\documentclass[12pt]{article}

\usepackage[latin1]{inputenc}
%\usepackage[danish]{babel}
\usepackage[danish, english]{babel}
\usepackage{a4}
\usepackage{moreverb}
\usepackage{epsfig,here}
\usepackage{subfigure}
\usepackage{rotate}

\newcommand{\efig}[4][\columnwidth]{
        \begin{center}
        \begin{figure}[h!t]
                \begin{center}
                \epsfig{file={#4},width={#1}}
                \caption{{#2} \label{#3}}
                \end{center}
        \end{figure}
        \end{center}}

\newcommand{\twofig}[8][\columnwidth]{
        \begin{center}
        \begin{figure}[h!t]
                \begin{center}
		\mbox{\subfigure[#5]{\epsfig{file={#6},width={#1}}}\quad
		      \subfigure[#7]{\epsfig{file={#8},width={#2}}}
                     }
                \caption{{#3} \label{#4}}
                \end{center}
        \end{figure}
        \end{center}}

\newcommand{\rotatefig}[4][\columnwidth]{
        \begin{center}
        \begin{figure}[h!t]
                \begin{center}
                \begin{sideways}
                \epsfig{file={#4},width={#1}}
                \end{sideways}
                \caption{{#2} \label{#3}}
                \end{center}
        \end{figure}
        \end{center}}


\begin{document}
%\efig[10cm]{Tilstandsdiagram for bestilling}{besttilst}{bestiltilst.eps}

\section{CPNTools interface}
\label{sec:ctinterface}

\subsection{Introduction}
This is a brief introduction to the interface elements and interaction 
techniques in CPNTools. After reading this, it should be possible
to explore CPNTools further, trying out the interaction techniques
on available net examples and tools. There are no detailed explanations 
of CP-nets herein, nor of the exact tools available in CPNTools; the 
latter can be found on the CPNTools help pages, located at 
\textit{http://wiki.daimi.au.dk:8000/cpntools-help/\_home.wiki}

\subsection{Workspace}
\label{sec:ctworkspace}
CPNTools runs in one big window, which can be maximized and resized
just as a standard window in a window-based environment. It consists 
of an \textbf{index} on the left side, showing text items (index nodes) 
that represent the accessible workspace elements, and a large open 
area, the \textbf{workspace}, for tools and net modules. (See 
fig.~\ref{fig:ctworkspace}). 

\efig[13cm]{CPNTools workspace}{fig:ctworkspace}{../screendumps/introduction/workspace_notes.ps}

\noindent
A net module, or \textbf{page}, is placed in a \textbf{sheet} when 
opened (one page per sheet), and the sheets are organized in 
\textbf{binders}. \textbf{Tabs} on the sheets make it possible to 
flip between several pages in one binder. The sheets can be moved 
around and placed in different binders, and one sheet can be in 
several binders, providing several views, e.g. with different zoom 
factors, on the associated page (see fig~\ref{fig:ctviews}). Binders 
can also be moved around and resized.

\efig[10cm]{Multiple views}{fig:ctviews}{../screendumps/introduction/multiple_views.ps}

\noindent
The workspace contains two cursors, one for the right hand and one
for the left. (Fig.~\ref{fig:ctstdcursors}) 

\efig[5cm]{Standard cursors}{fig:ctstdcursors}{../screendumps/introduction/twocursors.ps}

\noindent
Left-handed users can switch devices (using the trackball in the 
right hand and the mouse in the left hand), and it is possible to
plug in extra input devices to work collaborately with other users
on the same machine. Each new input device will have its own cursor.

\subsection{CP-net objects}

The pages contain the objects in the net, i.e. places, transitions,
arcs, inscriptions, etc.
\efig[10cm]{CP-net objects}{fig:cpnetobjects}{../screendumps/introduction/cpnetobjects.ps}

\subsection{Interaction techniques}

\subsubsection{Palettes}
\label{sec:ctpalettes}
Tool palettes consist of a collection of \textbf{tool cells}, each
containing an icon that indicates the tool in the cell. Tools are
located in \textbf{tool binders}.\newline
\noindent
Tool palettes are on sheets, and they can be placed together in 
binders (fig~\ref{fig:cttoolbinder}). This way, users can collect 
tools that they need to switch between for a certain task. 

\efig[6cm]{Tools in a binder}{fig:cttoolbinder}{../screendumps/introduction/toolbinder.ps}

\noindent
Toolbinders only contain tool-sheets, and cpnbinders only contain 
page-sheets; thus, it is not possible to put toolglasses and net pages
together in a binder.\newline

\noindent
Tools are selected from palettes by clicking on a cell with the left 
mouse button. The tool is now ''in the hand'' and clicking with the
left mouse button on an appropriate object will apply the tool. The 
cursor changes to indicate which tool is currently in the hand (see
fig.~\ref{fig:ctcursors}).\newline
\efig[10cm]{Examples of cursors}{fig:ctcursors}{../screendumps/introduction/cursorexamples.ps}

\noindent
While a tool is in the hand, it is still possible to move objects
around; this is done with a so-called \textbf{long click}, where
the mouse is pressed and held over an object, until the cursor
changes to indicate that the move tool is now active (see 
figure~\ref{fig:movewithtool}). When the move operation is finished, 
the user releases the mouse, and the tool that was in the hand before 
the move is back (indicated by a switch back to the associated tool 
cursor).

\twofig[4cm]{4cm}{Moving a binder with a tool in the hand}{fig:movewithtool}{Cursor for color tool}{../screendumps/introduction/before_move.ps}{Cursor changed to move cursor}{../screendumps/introduction/after_move.ps}

\paragraph{Icons}
Icons in the marking menus and palettes show which action is associated 
with the tool. An icon is either a graphical object or a text item. 
Marking menus currently use text items.

\noindent
An example of a graphical icon: the Transition tool has a small version 
of a transition as its icon, and the Change Color tool has a small 
colored triangle. The graphical icons change if the associated action 
does, e.g. if the Transition tool changes to make red transitions of a 
particular width and breadth, the icon turns into a red transition of 
the right proportions. (Fig~\ref{fig:deftrans}).

\twofig[6cm]{6cm}{Changing icons for transition tool}{fig:deftrans}{Black}{../screendumps/introduction/def_transition_black.ps}{Red, smaller}{../screendumps/introduction/def_transition.ps}

\noindent
Icons in palettes highlight when they are active, i.e. when a tool is
picked from a palette into the hand, the corresponding tool cell has
a darker background to indicate that it is active. (See 
fig.~\ref{fig:ctactivecell}).

\efig[5cm]{A highlighted cell}{fig:ctactivecell}{../screendumps/introduction/cell_highlight.ps}

\subsubsection{Toolglasses}
Toolglasses are semi-transparent, floating palettes (fig.~\ref{fig:tg}).

\efig[7cm]{A toolglass}{fig:tg}{../screendumps/introduction/../screendumps/introduction/tooglass.ps}

\noindent 
With toolglasses, users specify a tool and a target in one action by 
clicking through a cell onto an object. One hand moves the toolglass 
around, and the other hand clicks through with the left button. Clicking 
the right button on the device that holds the toolglass drops the 
toolglass as a fixed palette.\newline

Both hands can pick up a toolglass. The most common practice is to 
work with the toolglass in the non-dominant (typically left) hand.

\subsubsection{Marking menus}
The marking menus are nearly opaque, circular, context-sensitive menus. 
The marking menus appear when the right mouse button is pressed and 
held down for a short while. It appears with the center at the mouse 
coordinates. Moving the mouse cursor over the entries make the entries 
highlight to show which entry is chosen. (See fig.~\ref{fig:ctentrymm}). 
Releasing the mouse button on top of an entry will perform the 
corresponding action.

\efig[6cm]{CPNTools marking menus}{fig:ctentrymm}{../screendumps/introduction/before_mmdelete.ps}

\noindent
When the user has learned the position of the individual entries, it is
no longer necessary to pop up the menu, and a quick gesture can now
perform the desired action. The user presses down the right mouse
button and with a quick gesture moves the mouse in the direction of
the menu entry (see fig.~\ref{fig:ctgesture}).\newline

\twofig[6cm]{6cm}{Gesture interaction}{fig:ctgesture}{Gesture to delete item}{../screendumps/introduction/mmdelete.ps}{Text feedback on correct gesture}{../screendumps/introduction/after_mmdelete.ps}

\noindent
The starting point of a gesture is the focus point for the following
action. When the user wants to e.g. create an object, the object appears
where the gesture starts.

\subsubsection{Two-handed manipulation}
\label{sec:cttwohanded}
The two hands can interact with objects in a number of different 
ways, working simultaneously either on their own or together. Two 
buttons on each hand have separate functions, determining the
interactions of the hands. The following describes the three ways 
the hands can work; separately, asymmetrically, and symmetrically.

\paragraph{Separate hand interaction}
The two hands can work separately, performing similar interactions such
as moving objects and picking up tools (toolglasses in the non-dominant 
hand, individual tools in the dominant hand). 
\newline
In the following, I will use ''right'' and ''left'' for dominant and
non-dominant; the operations are reversible for left-handed users.
Possible separate hand operations are:
\begin{description}
\item[Move objects] By pressing the \textbf{left} button on the
mouse or trackball and dragging, users can move objects around. The 
cursor changes to the move cursor (see fig.~\ref{fig:movecursor}) to 
indicate when the object can be moved.
\efig[4cm]{The move cursor}{fig:movecursor}{../screendumps/introduction/after_move.ps}
Objects that can be moved are: \textbf{Pages} (by dragging the page 
tab), \textbf{binders} (by dragging the lightgrey title area),
\textbf{tool palettes} (by dragging the tab on the palette),
\textbf{CP-net objects} (by dragging the object).
\item[Pan] By pressing and dragging with the \textbf{left} button
on the mouse or trackball in the background of a page (i.e. the
light brown area), users can pan the page. The pan cursor (see
fig.~\ref{fig:pancursor}) indicates when the page can be panned.
\efig[4cm]{The pan cursor}{fig:pancursor}{../screendumps/introduction/pancursor.ps}
\item[Switch between pages and tools] By clicking on a tab (either
a page or a tool tab), users can bring a page or tool palette to 
front.
\item[Expanding/collapsing index triangles] By clicking on the
small triangles in the index (and on declaration sheets), users
can open and close the index entries. The triangles flash briefly
in green when they expand/collapse.
\item[Drag from index] By pressing and dragging on a text item
in the index, users can drag the item out to open a page, a
tool palette, or a declaration. Not all text items can be dragged
from the index.
\item[Pick up a tool] By clicking with the \textbf{left} button
of the mouse on a tool in a tool palette, users can pick up tools
in the hand. Note that this only works when the tool palette is
in palette mode, i.e. it is not possible to pick tools up from
a toolglass. 
A yellow frame around the tool cell indicates that the tool is 
selected (see fig.~\ref{fig:ctactivecell}). Clicking again on the 
selected tool will drop the tool, and the hand is now empty. When
there is a tool in the hand, the cursor shows which tool it is
(see fig.~\ref{fig:ctcursors}).\newline
While a tool is in the hand, it is still possible to move objects,
pan, bring pages to front, etc. This is then done with a ''long
click'', i.e. by pressing and holding down the mouse button until
the cursor changes to indicate that e.g. an object can be moved.
When the operation is done, the cursor returns to the tool cursor,
and 
\item[Apply a tool] When a tool is in the hand, it can be applied
by clicking on an appropriate object with the \textbf{left} mouse
button. If there is no tool in the hand, a toolglass tool can be
applied by clicking through the cell with the \textbf{left} mouse
button onto an appropriate object. Fig.~\ref{fig:application} shows
how to change colors on an object with a tool in the hand and with
a toolglass.
\twofig[6cm]{6cm}{Coloring a place}{fig:application}{Tool in hand: Click on place}{../screendumps/introduction/tool_color.ps}{Toolglass: Click through on place}{../screendumps/introduction/toolglass_color.ps}
\item[Pick up toolglass] Clicking with the \textbf{right} trackball
button on a tool binder picks the binder up as a toolglass, i.e. a
semi-transparent palette that sticks to the trackball. The right 
hand apply tools in the toolglass to objects underneath as described 
above.
\item[Drop toolglass] If the palette is in toolglass mode, a 
click with the \textbf{right} trackball button drops it on the
background (as a palette), and it no longer sticks to the trackball. 
\item[Move toolglass] While the toolglass sticks to one of the 
input devices, it can be moved around by moving the device, thus 
keeping tools close to where the user works. 
\end{description}

\paragraph{Asymmetrical two-handed interaction}
The two hands can perform tasks together, working in two different 
ways. These tasks are:
\begin{description}
\item[Apply tool from toolglass] While the trackball moves the toolglass
around, the mouse clicks through the cells to apply the tools, as
described above.
\end{description}

\paragraph{Symmetrical two-handed interaction}
The two hands can work together, symmetrically, on the same task:
\begin{description}
\item[Zoom] To zoom in on a page, position the trackball cursor
somewhere on the page background (i.e. the light brown area), and
press the \textbf{left} trackball button. Now position the mouse 
cursor somewhere on the same page background, and press the 
\textbf{left} mouse button. Moving the cursors closer and farther 
from each other (''stretching'' and ''shrinking'') will zoom in and 
out on the page. There is no cursor feedback on the zoom; only the 
left cursor changes during this interaction.
\item[Resize binder] To resize a binder, position the trackball 
cursor somewhere on the binder's title area (i.e. the light grey 
bar in the top of the binder), and press the \textbf{left} trackball
button. Now position the mouse cursor somewhere in the binder, e.g. 
on the background of the page in the binder, and press the 
\textbf{left} mouse button. Moving the cursors closer and farther 
from each other will resize the binder. There is no cursor feedback 
on this; only the left cursor changes during this interaction.
\end{description}
The interactions for zoom and resize are the same, the only difference
is where to position the trackball cursor. 

\label{endrapport}

\end{document}